\documentclass[twoside]{article}

\usepackage[accepted]{aistats2021}
% If your paper is accepted, change the options for the package
% aistats2021 as follows:
%
%\usepackage[accepted]{aistats2021}
%
% This option will print headings for the title of your paper and
% headings for the authors names, plus a copyright note at the end of
% the first column of the first page.

% If you set papersize explicitly, activate the following three lines:
%\special{papersize = 8.5in, 11in}
%\setlength{\pdfpageheight}{11in}
%\setlength{\pdfpagewidth}{8.5in}

% If you use natbib package, activate the following three lines:
%\usepackage[round]{natbib}
%\renewcommand{\bibname}{References}
%\renewcommand{\bibsection}{\subsubsection*{\bibname}}

% If you use BibTeX in apalike style, activate the following line:
%\bibliographystyle{apalike}

\begin{document}

% If your paper is accepted and the title of your paper is very long,
% the style will print as headings an error message. Use the following
% command to supply a shorter title of your paper so that it can be
% used as headings.
%
%\runningtitle{I use this title instead because the last one was very long}

% If your paper is accepted and the number of authors is large, the
% style will print as headings an error message. Use the following
% command to supply a shorter version of the authors names so that
% they can be used as headings (for example, use only the surnames)
%
%\runningauthor{Surname 1, Surname 2, Surname 3, ...., Surname n}

\twocolumn[

\aistatstitle{A New Method for Discovering Daily Depression from Tweets to Monitor Peoples Depression Status Analysis
}

\aistatsauthor{ Omolola Solaru  }

\aistatsaddress{ Georgia State University,Fall 2020,CSC 4350} 

]
\begin{abstract}
Through the utilization of traditional media for the simple purpose of communicating it has become age-old in the wake of advancing Social Media, and there is enormous development as a result of simplicity in innovation, cost, multimedia facility, multiple device access, and faster reach\cite{tusharazhang}. Online Media turned into a methods for communications for association among individuals known and unknown \cite{tusharazhang}. It is utilized to advance business, computerized marketing, learning information, and a lot more areas. Individuals will in general share their feelings, and web-based media became one such asset that numerous individuals are pulled in to share their perspectives on an item, area, and themselves as well. They update their status, articulation of their emotions/sickness in online media that has become a routine for some people. The tendency for posting information in web-based media encouraged researchers to gather, decipher, and examine the data. The main issue is discovering a highly effective method that allows for the monitoring of people who shows signs of depression through their social media posts, in order to prevent suicide and/or the harming of other individuals. The method used to replicated and demonstrate a situation like this were by identifying seven keywords that's corresponds with Kessler's \cite{tusharazhang}. ten-point questionnaire, which is the most utilized technique to discover the individual Psychological Distress scale. They were able to acquire the keyword related tweets from twitter.com using the assistance of API. These tweets utilizing #depress, #failure, #hopeless, #nervous, #restless, #tired, #worthless were gathered on a random day. From there they processed the quantity of antonym tweets that are available in the anticipated Kessler keyword tweets had predicted positive but are negative FN from the Kessler keyword tweets. Additionally, they determined the Kessler keyword related tweets in antonym Kessler tweets predicted negative however was positive FP. Table-1 \cite{tusharazhang} in the article  gives the genuine number of tweets in every category of that table.
    
\end{abstract}



\section{Critique}
The overall paper is exemplary in its findings for the procedures used to demonstrate the average quantity of tweets containing keywords corresponding to depression based on the 
Kessler's Psychological Distress Scale. Due to these findings this contributes to suicide and harm prevention. Additionally,
since no other work has been really done to really analyze the duration that has happiness is compared to a day using social media, it interesting to see this article take it a step further with its experiment procedures. The only issue that is challenging is the fact that during the analysis of the experiments placed in the article they are some terms that were confusing and not really explained further enough in the importance they have to the results in the table.


\section{Synthesis}
Additionally,a way that can further develop the research  would examine more live streams that happen on social media platforms and attempt to detect signs of depression in an individual  though there facial expressions,speech pattern, and average Kessler's keyword during speech. From this an AI can be developed and used to analyze the individuals behavior patterns during live streams. 

\bibliographystyle{ieeetr}
\bibliography{reference}

\end{document}
